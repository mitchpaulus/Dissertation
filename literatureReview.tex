%%%%%%%%%%%%%%%%%%%%%%%%%%%%%%%%%%%%%%%%%%%%%%%%%%%
%
%  New template code for TAMU Theses and Dissertations starting Fall 2012.  
%  For more info about this template or the 
%  TAMU LaTeX User's Group, see http://www.howdy.me/.
%
%  Author: Wendy Lynn Turner 
%	 Version 1.0 
%  Last updated 8/5/2012
%
%%%%%%%%%%%%%%%%%%%%%%%%%%%%%%%%%%%%%%%%%%%%%%%%%%%

%%%%%%%%%%%%%%%%%%%%%%%%%%%%%%%%%%%%%%%%%%%%%%%%%%%%%%%%%%%%%%%%%%%%%%%
%%%                           SECTION II
%%%%%%%%%%%%%%%%%%%%%%%%%%%%%%%%%%%%%%%%%%%%%%%%%%%%%%%%%%%%%%%%%%%%%%

\chapter{\texorpdfstring{\MakeUppercase{Literature Review}}{Literature Review}}


A significant amount of research has been completed in the field of
control optimization schemes. Several different approaches including
direct search methods, non-linear programming, genetic algorithms, and
artificial neural networks have been suggested. Optimizations have been
completed at all levels of the HVAC system: Zone, AHU, and Plant.  

When discussing optimization, it is important to be clear on what is
meant by \textit{Controls Optimization}. When discussing the subject
with a controls engineer, they will likely be analyzing parameters such
as the settling time, overshoot, and rise time --- dynamic parameters.
This research is focused more on the optimization of the
\textit{steady-state} behavior of the system. 



%    _____      __              _       __     ____        __ 
%   / ___/___  / /_____  ____  (_)___  / /_   / __ \____  / /_
%   \__ \/ _ \/ __/ __ \/ __ \/ / __ \/ __/  / / / / __ \/ __/
%  ___/ /  __/ /_/ /_/ / /_/ / / / / / /_   / /_/ / /_/ / /__ 
% /____/\___/\__/ .___/\____/_/_/ /_/\__/   \____/ .___/\__(_)
%              /_/                              /_/           


\section{Setpoint Optimization}

Since the time constant of AHU dynamics is on the order of minutes and
the time constant of the thermal loads on the building are on the order
of minutes to hours, it is appropriate to focus on the steady state
behavior of AHUs \cite{Bourdouxhe1998}.

Ke and Mumma were one of the first to attempt to balance the benefits of
raising the supply air temperature with the penalties associated with
increased fan energy and decreased dehumidification potential
\cite{Ke1997OptimizedSystems}. This paper was limited in the fact that
it made no discussion on actual implementation. The simulated results
showed that optimizing the balance between fan, cooling, and reheat
energy resulted in approximately 6\% energy savings annually versus a
fixed supply air temperature setpoint. They showed that the benefit was
most promising in the more temperate weather, when the VAV system was
not running at minimum primary airflow. The proposed work would not only
be optimizing the supply air temperature, but also the fan static pressure setpoint.   

Wang and Song also balanced supply air temperature with fan power and
included economizer control \cite{Wang2012AirCycles}. They found energy
savings reaching up to 90\% under the specific outdoor air conditions
and space loads in their simulation. The universal control sequence they
proposed relied on an outside air temperature sensor, supply air
temperature sensor, and supply air flow sensor. A limitation of this
work was the assumption that the terminal units did not have reheat. It
is well known though that terminal unit reheat energy can be a
significant contributor to energy consumption and the reduction of
reheat is often the source of significant energy and money savings in
existing building commissioning.  

Qin completed a dissertation in 2014 entitled, ``A Data-Driven Approach
for System Approximation and Set Point Optimization, With a Focus in
HVAC Systems'' \cite{Qin_2014_Res_Letters}. The focus of this work was
related to the programming of thermostats in residential homes and did
not cover any buildings in the commercial sector. The control responses
for the rooms were being optimized and not the temperature setpoints of
the rooms.  

Huh and Brandemuehl optimized the setpoint for hot and humid climates
\cite{Huh2008}. The most significant limitation of this work was that it
was heavily focused on a DX unit that would be common to a retail or
supermarket and was also focused on only a hot and humid climate.
Engdahl and Johansson also devoted investigation into the optimization
of the supply air temperature setpoint in a VAV system
\cite{Engdahl2004}, but only studied AHUs with 100\% outside air. 

While the focus of this proposal is on air handling units, the research
regarding the optimization of central plants should not be ignored
\cite{Ahn2001, Yu2008OptimizingChillers}. At the plant level, Braun has
completed a great deal of research \cite{braun1990, Braun2007RP1252,
Braun2007HybridCooling, Braun2007GeneralControl}. He has investigated
other optimizations, such as ones for night precooling with packaged
rooftop systems \cite{Braun2005}. Crowther and Furlong showed that
advanced optimization controls for chiller-tower condenser-pump systems
could make a 5-8\% improvement in annual energy use
\cite{Crowther2004}. Henze has also been active in
optimizing thermal storage systems \cite{Henze2005} along with
Kintner-Meyer and Emery \cite{KintnerMeyer1995}. Lu et al. also focused
on the optimization of the entire HVAC system, plant and building, in a
series of papers. \cite{LuLu2004, LuLu2005Part1, LuLu2005Part2,
LuLu2005HVACSystemOptimization}. 

United States patents have even been written on the topic of HVAC
optimization. Cascia has a patent on optimal control for a cooling and
heating plant with DDC control \cite{Cascia1999} and Seem has a patent
that describes the strategy to optimally control an air side economizer
\cite{Seem2002}. 

Numerous other researchers have thoroughly covered many aspects of an
optimally operating HVAC system, which is not limited to
\cite{Gruber2014AlternativeBuildings,Cho2009Single-ductOptimization,Nassif2005,Zaheer-uddin2000OptimalBuildings,Wang2000Model-basedAlgorithm,Henze2003EvaluationSystems,Liu2006ExperimentalFoundation,
Zheng1996,
Ning2010Neuro-optimalSystem,Atthajariyakul2004,Cui2004,XuXinhua2009,SunZhongwei2011,Mumma1997EnergyControl,Mossolly-Ghali-Ghaddar_2009_Energy}.
Wang and Ma provided a comprehensive review of supervisory control
research \cite{Wang2008}. This literature is limited in
regards to efficient methods of rapid implementation. The proposed
research also presents a different optimization methodology that focuses
on the use BAS trend data paired with small, calibrated, first-principle
models. 



%%%%%%%%%%%%%%%%%%%%%%%%%%%%%%%%%%%%%
%% Advanced Computation Techniques %%
%%%%%%%%%%%%%%%%%%%%%%%%%%%%%%%%%%%%%



\section{Advanced Computation Techniques and Controls}

The HVAC industry has slowly begun to employ the advances in computer
science. Researchers from the University of Iowa (Kusiak, Li, Xu, Tang,
Wei) have published a great deal of work on the optimization of all
types of HVAC systems. They have applied data mining algorithms and
computational intelligence algorithms to data-driven optimization for
the cooling output of air handling units and plants
\cite{Kusiak2014MinimizationOfEnergyConsumptionInHVAC, HeXiaofei2014,
Kusiak2013MinimizingEnergyConsumption,
Kusiak2012ModelingAndOptimizationOfHVAC, Kusiak2011MultiObjective,
Kusiak2010ReheatBox, WeiXiupeng2015,
WeiXiupeng2014ModelingAndOptimizationOfAChillerPlant, Kusiak2010,
Kusiak2010ModelingAndOptimization,
Kusiak2011OptimizationOfAnHVACSystemWithAStrength}. They have used
techniques like neural networks, evolutionary type programming, and
multi-perceptron ensembles. They have also applied
\textit{particle-swarm optimization} in several of their publications. 

The use of genetic algorithms or general evolutionary programming
techniques has been applied in several pieces of research. Genetic
algorithms have been used to optimize chilled water supply temperature,
supply air temperature, fan control, and outdoor air control, related to
many different kinds of systems, including variable air volume and
variable refrigerant volume
\cite{Fong2006HVACProgramming,Jin2005Prediction-basedSystems,Parameshwaran2010EnergyAlgorithm,Congradac2009HVACAlgorithms}

\subsection{Model Predictive Control}

Model predictive (or receding-horizon) control (MPC) has been a popular
research field in control theory and has been successfully implemented
in practice. Li et. al \cite{Li2015} recently showed the
benefits of MPC in both simulation and in experimental work. They
estimated electrical consumption savings to be 18\% for a 75,150
ft\(^2\) building in Philadelphia during a week in August, and found
that in 75\% of their 20 test days they had energy consumption savings
of over 20\%. They also used a centralized architecture where BAS data
was passed through a middleware with a historical database to Matlab and
an AMPL optimization system, with results of that system dynamically changing
the building HVAC system.

% The citations here are coming directly from Afram 2017.
Afram et al. also studied the combination of ANNs and MPC
\cite{Afram2017}. As described in \cite{Afram2017}, the combination of
these two techniques has been used for the following different control
objectives:

\begin{enumerate}
    \item Minimize energy consumption \cite{Ferreira2012, Huang2015a, Kusiak2011OptimizationOfAnHVACSystemWithAStrength, Kusiak2014MinimizationOfEnergyConsumptionInHVAC,WeiXiupeng2015, Garnier2015, Kim2016, LuLu2005HVACSystemOptimization, Ning2010Neuro-optimalSystem}
    \item Maintain thermal comfort \cite{Ferreira2012, Kusiak2011OptimizationOfAnHVACSystemWithAStrength, Kusiak2014MinimizationOfEnergyConsumptionInHVAC, WeiXiupeng2015, Garnier2015, Kim2016}
    \item maintain indoor air quality (IAQ) at an acceptable level \cite{Kusiak2011MultiObjective}
    \item Minimize operating cost \cite{Garnier2015, Lee2015, Huang2015a, Ruano2015, Ruano2016}
    \item Maintain visual comfort at an acceptable level \cite{Kim2016}
    \item Minimize retrofit cost \cite{Asadi2014a}
    \item Minimize thermal discomfort hours \cite{Asadi2014a}
\end{enumerate}

This proposal has a focus on steady state behavior, but the dynamic
behavior of buildings and its controls are also important. Seem has
published research comparing a finite state machine (FSM) sequencing to
the more common split-range sequencing control logic \cite{Seem1999}.
Xu, Li, and Cai proposed a receding-horizon optimization control that
uses a typical PID type controller \cite{XuMin2005}. This work had a
focus on practicability in that it required no changes in the hardware
or  the definitions of the common control parameters related to a PID
controller. Yuan and Perez used a model-predictive controller in order
to control temperature and ventilation for multiple zones
\cite{Yuan2006Multiple-zoneStrategy}. Freire, Oliveira, and Mendes also
used predictive controllers for thermal comfort optimization
\cite{Freire2008PredictiveSavings}.  Guo, Song, and Cai investigated
neural networks in HVAC control \cite{Guo2007}.

\begin{table}[ht]
\caption{Techniques applied in HVAC system optimization.}
\label{tab:Techniques}
\begin{tabular}{L{8cm} p{5cm}}
\toprule
Techniques 			&       Sources	 	\\
\midrule \midrule
Quadratic Least Square Regression	&  \cite{Ahn2001}  \\

Gradient Based Optimization, Golden Complex Search  & \cite{Huh2008}\cite{Atthajariyakul2004} \\

% Control Adjustments or Optimizations from Models/Data 	& 	\cite{Cho2009Single-ductOptimization}\cite{Braun2007GeneralControl}\cite{Braun2007RP1252}\cite{Braun2007HybridCooling}\cite{Braun2005}\cite{Crowther2004OptimizingTowers} \\ 

% & \cite{Engdahl2004}\cite{Ke1997OptimizedSystems}\cite{KintnerMeyer1995}\cite{LuLu2005Part1}\cite{LuLu2004}\cite{Mumma1997EnergyControl}\cite{SunZhongwei2011} \\

% &  \cite{Wang2012AirCycles}\cite{Yu2008OptimizingChillers}\cite{Zaheer-uddin2000OptimalBuildings}\cite{Zheng1996} \\


Genetic Algorithms 	& \cite{LuLu2004}\cite{LuLu2005Part2}\cite{LuLu2005HVACSystemOptimization}\cite{Nassif2005}\cite{Wang2000Model-basedAlgorithm}\cite{XuXinhua2009} \cite{Mossolly-Ghali-Ghaddar_2009_Energy}\cite{Congradac2009HVACAlgorithms}\cite{Jin2005Prediction-basedSystems}\cite{WeiXiupeng2014ModelingAndOptimizationOfAChillerPlant}	\\


Analytical Linear Optimization 	& 	\cite{Cui2004} \\

Evolutionary Programming    & 	\cite{Fong2006HVACProgramming}\cite{Kusiak2011MultiObjective}	 \\

Evolutionary Strategy       &  \cite{Kusiak2010}  \\

Model Predictive Control/Receding Horizon    &  \cite{Henze2005}\cite{Gruber2014AlternativeBuildings}\cite{Freire2008PredictiveSavings}\cite{Kusiak2010ReheatBox}\cite{XuMin2005}\cite{Yuan2006Multiple-zoneStrategy} \\

Neural Networks     &  \cite{Ning2010Neuro-optimalSystem}\cite{Kusiak2010}\cite{Guo2007}\cite{Kusiak2012ModelingAndOptimizationOfHVAC}\cite{Kusiak2011OptimizationOfAnHVACSystemWithAStrength}\cite{Kusiak2014MinimizationOfEnergyConsumptionInHVAC}	 \\

Particle Swarm Optimization   & \cite{HeXiaofei2014}\cite{Kusiak2010ReheatBox}\cite{Kusiak2010ModelingAndOptimization}\cite{Kusiak2012ModelingAndOptimizationOfHVAC}\cite{Kusiak2011OptimizationOfAnHVACSystemWithAStrength}\cite{WeiXiupeng2015}\cite{WeiXiupeng2014ModelingAndOptimizationOfAChillerPlant}\\

Harmony Search    &  \cite{HeXiaofei2014}   \\ 

Model Free Reinforcement Learning & \cite{Henze2003EvaluationSystems}\cite{Liu2006ExperimentalFoundation} \\

ARMA   &  \cite{Jin2005Prediction-basedSystems}  \\

Data Mining Algorithms & \cite{Kusiak2011MultiObjective}\cite{Kusiak2010}\cite{Kusiak2010ReheatBox}\cite{Kusiak2010ModelingAndOptimization} \\ 

Multiple-Linear Perceptron Ensemble & \cite{Kusiak2014MinimizationOfEnergyConsumptionInHVAC}\cite{Kusiak2010ModelingAndOptimization}\cite{WeiXiupeng2015}\cite{Kusiak2013MinimizingEnergyConsumption} \\ 

Interior Point Method & \cite{Kusiak2014MinimizationOfEnergyConsumptionInHVAC} \\

Adaptive Neuro-Fuzzy Inference Systems (ANFIS) & \cite{LuLu2005HVACSystemOptimization}  \\

Genetic Fuzzy Optimization &  \cite{Parameshwaran2010EnergyAlgorithm}  \\ 

Finite State Machine &  \cite{Seem1999} \\ 

\bottomrule
\end{tabular}
\end{table}




% \section{AHU Setpoint/Control Strategy Optimization}

% Mumma has also optimized the ventilation control in a variable air volume system \cite{}.


% Mossolly, Ghali, and Ghaddar examined advanced control strategies based on comfort and indoor air quality \cite{}. One proposed strategy explored adjusting the fresh air rate and the supply air temperature, maintaining a temperature setpoint in the zones and assuring indoor air quality. The second strategy also controlled the fresh air rate and supply air temperature but attempting to maintain thermal comfort based on the predicted mean vote of the zone. They found that they saved 30.4\% in operational costs based on simulations of a building in Beriut during the cooling summer season of 4 months with the control strategy based on the predicted mean vote.

% The work by Mossolly is an example of an advanced, optimized control strategy that does not make much mention of the feasibility of implementation. The strategy is limited by the number of weighting factors in the objective function, and the use of sometimes unreliable \(\text{CO}_{2}\) sensors for indication of demand. 


% Many of these approaches have significant limitations making them difficult to implement and scale. Some of the model-based optimizations include parameters unavailable in typical building BAS systems and the uncertainty in these parameters can be significant. Model specification would be necessary for each building setup, requiring significant engineering work. 

% It is desired to eliminate these limitations by developing a method that focuses only on the most robust sensors typical in a CAV/VAV reheat system, uses simple engineering formulas, and requires very little engineering setup. 
% The optimization literature focuses heavily on the methods and less on the implementation. This work would investigate how a centralized, protected, high-level optimized control program could be used and scaled to numerous buildings that each may have different control vendor programs.  

%     ____  ___   _____    ______                                    
%    / __ )/   | / ___/   / ____/___  ____ ___  ____ ___  __  ______ 
%   / __  / /| | \__ \   / /   / __ \/ __ `__ \/ __ `__ \/ / / / __ \
%  / /_/ / ___ |___/ /  / /___/ /_/ / / / / / / / / / / / /_/ / / / /
% /_____/_/  |_/____/   \____/\____/_/ /_/ /_/_/ /_/ /_/\__,_/_/ /_/ 
                                                                                                                               


\section{BAS Communication}

There are many different networking and communication levels and
protocols related to buildings. Kastner, Neugschwandtner, and Soucek et
al. provided a summary of the different systems that exist in buildings
\cite{Kastner2005}. An understanding of the different levels (described
as management, automation, and field by \cite{Kastner2005}) is important
for developing any automated system. The most important open systems
related to building automation include BACnet, LonWorks, EIB/KNX. Other
important and relative standards, protocols, and technologies are shown
in \tableref{} \ref{tab:CommunicationProtocols}.

\begin{table}
\centering
\begin{tabular}{ c }
\toprule

Network Communication Protocols \\
\midrule
\midrule
BACnet\\ 
LonWorks \\
EIB/KNX\\
IP \\
\\
\midrule
Object Access Protocols \\
\midrule
\midrule
Common Object Request Broker Architecture (CORBA) \\
Java Remote Method Invocation (RMI)     \\
Microsoft Distributed Component Object Model (DCOM)  \\
Simple Object Access Protocol (SOAP) \\
\\
\midrule	
Architecture Style \\
\midrule
\midrule
Representational State Transfer (REST) \\
\\
\midrule
Local Area Network Type \\
\midrule
\midrule
Ethernet \\
ARCNET  \\
Master-Slave/Token-Passing (MS/TP) \\
LonTalk  \\
Point-to-Point (PTP)  \\
\\
\midrule
Data Format \\
\midrule
\midrule
Javascript Object Notation (JSON) \\ 
Extensible Markup Language (XML)  \\
\bottomrule
\end{tabular}
\caption{Important Standards/Technologies in Building Automation}
\label{tab:CommunicationProtocols}
\end{table}


Project Haystack is an important open source initiative that is looking
to bring specific naming conventions to building operational data
(project-haystack.org). It accomplishes this at a more abstract level
than BACnet, LonWorks, and EIB/KNX, using lexical rules. Before data
from sensors in buildings can be used effectively, an ``association''
of the data to an HVAC object (say data being associated with an air
temperature sensor located in a specific air handling unit) needs to
occur. In order to create an environment where an intelligent building
makes sense, raw numeric data needs to have extensive meta-data to give
it context. This can be related to engineering units, what piece of
equipment the data belongs to, the hierarchy of equipment and
relationships to one another, and other modifiers. 



%%%%%%%%%%%%%%%%%%%%%%%%%%%%%%%%%%%%%%%%%%%%%%
%% SUMMARY OF LITERATURE SECTION %%%%%%%%%%%%%
%%%%%%%%%%%%%%%%%%%%%%%%%%%%%%%%%%%%%%%%%%%%%%
\section{Summary of Literature}

Not surprisingly, the goal of any controlled system is to operate
optimally. It is to be  expected that there would be a wealth of
literature on the optimization of all different components in a
particular HVAC system. Any optimization problem has objectives and
constraints. Many different objective functions and constraints to
optimization problems in HVAC have been proposed. Numerous different
optimization techniques and methods have been implemented and analyzed
by different researchers. 

However, there are several limitations and deficiencies in the
optimization literature. Not all authors have focused on the scalability
of different optimization methodologies. Presently it is not feasible to
implement complicated data-mining algorithms on every individual BAS or
controller, partly due to installation time and partly due to the lack
of facilities managers that have the necessary training to understand
the algorithms and that can keep the system running properly. Facility
managers already have difficulties in maintaining the most basic HVAC
systems. 

Many of the techniques proposed in the literature also are dependent on
sensors or information that are currently unavailable in typical
commercial HVAC set-ups. For large-scale implementation, a methodology
must use the minimum number of trends to complete useful optimization.

For an individual system, there are more than enough well documented and
effective optimization methods and algorithms. Yet very few buildings in
practice have these optimizations in place because the other numerous
challenges have not been fully solved. These challenges include
complexity in implementing with different BAS vendors, training a staff
to understand the logic behind the optimizations, and simply the
technician effort to install and setup implementations.

This research proposes its own intuitive optimization method based on
small first-principle models using historical data to calibrate that is
designed to efficient in scaling. The research investigates how to use
current communication protocols with BAS systems and implementable
methodologies to apply a single set of optimization logic and code to
many and varied air handling units. 

\section{Typical Trends Available from Commissioning Projects}

Unfortunately, at the current time, it is uncommon to have all the
explicit sensors available for system identification of HVAC equipment.
It is rare to have the capability to sub-meter the energy use of all the
individual components including the fans, cooling coils, and heating
coils. 

\subsection{Common Case}

The following suggestions are based on the data for over \num{150,000} trends
stored in Implementer along with personal experience. The statistical
results presented are from until April 13, 2016. The total
number of configured single duct AHUs at this time was 846. The number
of trends that existed in these 846 containers was 11,806, or
approximately 14 trends per AHU. Table \ref{tab:PointBreakdown} gives
the percentage breakdown of the types of points that we have seen in
current Implementer projects. Note that due to particular Implementer
considerations, the occurrence percentage should not be evaluated
directly (due to some projects not being properly set up all), The
relative relationship between trend types is the more significant
result.  


\begin{table}
\caption{Breakdown of points typically available in single duct AHU systems currently in Implementer.}
\label{tab:PointBreakdown}
\small
\begin{tabular}{L{3cm} L{1.2cm}}
    \parbox[b][][b]{2.9cm}{Point Type} & Occurrence \\
\toprule
65\%+                                  &            \\
\midrule
DAT/SAT                                & 83\%       \\
CHW Valve                              & 73\%       \\
Duct Static Pressure                   & 67\%       \\
\midrule
45\% - 65\%                            &            \\
\midrule
Outdoor Air Damper                     & 65\%       \\
Return Air Temp                        & 58\%       \\
DAT/SAT Setpoint                       & 53\%       \\
Return RH                              & 52\%       \\
Fan Status Points                      & 50\%       \\
Mixed Air Temp                         & 48\%       \\
Return Air CO\textsubscript{2}         & 48\%       \\
\midrule
25\%-45\%                              &            \\
\midrule
Filter \(\Delta\)P                     & 35\%       \\
Static Pressure Stpt.                  & 35\%       \\
Outdoor Air Flow                       & 28\%       \\
Return Air Damper                      & 27\%       \\
Occupied Status                        & 25\%       \\
\end{tabular}
\begin{tabular}{L{3.1cm} r}
\midrule
5\% - 25\%         &      \\
\midrule
Space Temp         & 21\% \\
Supply Air Flow    & 15\% \\
Fan Status         & 15\% \\
Fan S/S            & 14\% \\
Outdoor Air Temp   & 13\% \\
Heating Coil Valve & 21\% \\
Modes              & 10\% \\
Fan Power          & 9\%  \\
Preheat Temp       & 9\%  \\
CCLT               & 8\%  \\
Mixed Air Damper   & 7\%  \\
CHW Supply Temp    & 6\%  \\
Return Air Flow    & 6\%  \\
CHW RT             & 6\%  \\
Air Changes        & 5\%  \\
Fan Proofs         & 5\%  \\
Limits             & 5\%  \\
\end{tabular}
\begin{tabular}{L{3.1cm} r}
\midrule
0\% - 5\%             &               \\
\midrule
\# Boxes In Reheat    & 4\%           \\
Misc. Alarms          & 4\%           \\
Supply Fan kW         & 4\%           \\
Economizer Status     & 3\%           \\
Space Humidity        & 3\%           \\
CCLT Setpoint         & 3\%           \\
\% Load               & 2\%           \\
Cool/Heat Coil Flows  & 2\%           \\
Runtimes              & 2\%           \\
Water Pressure        & 1\%           \\
Fan Volts             & 1\%           \\
HW	 Supply Temp      & \textless 1\% \\
HW Return Temp        & \textless 1\% \\
Fan Current           & \textless 1\% \\
Outdoor Air Flow Stpt & \textless 1\% \\
\end{tabular}
\end{table}

If the building in question consumes district chilled and hot water for
thermal HVAC processes, the importance of additional trends is lessened
since it is clear that the chilled water is used to meet the cooling
load of the building and similarly for the hot water. The electricity
use is then left to lights and equipment, along with the fan energy.

However, if natural gas and electricity are the only energy supply types
for the building, additional information coming from trend data becomes
more important to disaggregate the end uses. 

It is mentioned in the problem that it can be assumed that monthly
utility bills are available. Having a smaller time interval on the
utility data can also aid significantly in the calibration. Daily data
can help aid in discriminating weekday/weekend profiles, while hourly
data can aid in exposing the diurnal cycle of the building. 


\subsubsection{Fan Power}

In all cases, having the fan power trended will aid in the electricity
use breakdown, as this directly supplies this particular energy end use.
In many cases, it is acceptable to combine lighting energy and the
non-HVAC related internal electricity use together. 

The fan power is trended directly on less than 10\% of systems in
Implementer. The next best option in 15\% of the cases is using the
supply flow along with manufacturers specifications to estimate
fan power. With the design flow values from mechanical drawings,
fractional power curves based on part-load ratio can be used to estimate
the fan energy use. 

The cooling energy that results from the chiller/HVAC equipment can be
estimated from the air-side energy balance across the cooling coils
(assuming that this building is not using individual fan coil units or
rooftop units). The cooling energy is related to the supply air flow,
\(T_{ma}\), and supply air temperature. It is also related to the mixed
air humidity. 


\subsubsection{Cooling Energy}
At the air handler level, the sensible cooling energy is related to the
supply flow, and the difference between the mixed air temperature, \(T_{ma}\),
and the supply air temperature, \(T_{sa}\). The supply flows can vary
significantly from one air handler to another, and also under different
loadings. 

\(T_{ma}\) is the lowest priority of the three air-side parameters for
estimating the sensible cooling energy in the AHU. A reasonable estimate
of the outdoor air temperature is available from local weather stations,
in addition to commonly trended outdoor air temperatures at the site
itself. The return air temperature in air handling uints are near the space
temperature setpoints. It is known that the \(T_{ma}\) must be between
the return and outdoor air temperature (if the airstreams are indeed
being mixed). The most robust estimate for the \(T_{ma}\) would be
halfway between these two measurements. 

As an argument for the claim that \(T_{ma}\) is the lowest priority of
the 3 parameters, the difference between \(T_{oa}\) and 72\(^\circ\)F
for the hourly outdoor air dry-bulb temperature weather data for College
Station during the year of 2015 was calculated. 1/2 of this
difference would be the worst case error estimation. The median of these
half-differences was 4.9\(^\circ\)F, and the mean was 6.4\(^\circ\)F.
These are robust estimates of the upper bounds on the error of the
estimation of \(T_{ma}\), with no guidance other than the assumption
that the \(T_{ma}\) is between the outdoor air temperature and return
air temperature. With any additional information regarding the outdoor
air fraction, the estimate would be even better. 

Determining the latent load across the cooling coil is difficult using
air-side parameters. Humidity sensors are traditionally unreliable and
two would be necessary to calculate an absolute humidity
difference.

In this sense, metering the water-side parameters would be a more
reliable estimator of the total cooling load, including the latent
effect. However, trended water-side sensors has not been seen in the 4
years of Implementer projects.

\subsubsection{Reheat Energy}

Unless the building has some other large hot water end uses, much of the
natural gas use will be directly related to heating/reheat. Electricity
use will then be distributed between lights and non-HVAC equipment,
fans, along with chiller/HVAC equipment. 

If the building is heated using natural gas, the natural gas consumption
can be an adequate indicator of the level of reheat use in the building.
At the current time, trending all the data from terminal units is
uncommon. 

Having the supply air temperature and supply air flow is important
because it not only helps fix the parameters for the cooling energy end
use at the AHU but also the parameters for estimating reheat in the
terminal units. If the terminal units are aggregated together, with the
knowledge of the total flow and supply air temperature, along with
potentially having measured natural gas use, the only parameter left is
the discharge temperature to the zones. 

\subsubsection{Prioritization} \label{sec:Prioritization}
\begin{enumerate}
\item Supply air temperature -- This parameter is crucial in estimating both the sensible cooling energy and the reheat energy of the system. 
\item Fan Power -- Directly returns the energy end use for fans.
\item Supply air flow -- A key parameter in estimating the flow both at the AHU and the total of which is going to the terminal units, affecting both the cooling and reheat energy end uses.
\item Outdoor air flow -- Important in fixing the ventilation load at the AHU.
\item Terminal unit flows -- Helps set the minimum primary air flow parameter, a sensitive parameter affecting the reheat. 
\item Terminal unit discharge temperatures -- Aids the estimation of
    zone reheat. Although this trend may not be as useful if the
    terminal units bring in plenum air for mixing. 
\item Space temperature -- Defines the zone temperatures. 
\item Mixed air temperature -- Aids in the determination of the sensible cooling load and estimation of the outdoor air fraction
\item Return air temperature -- Aids in the estimation of the outdoor air fraction or ventilation load.
\item VSD speed, VFD frequency, etc. -- Indicator of the part-load ratio for the equipment, which may be used in the estimation of fan energy. 
\item Preheat Temperature -- Sets this parameter in the model, aiding the estimation of the heating end uses.
\end{enumerate}

Sensors that may be of medium usefulness:

\begin{itemize}
\item Fan Status or occupied/unoccupied status -- Can give indication about the runtimes and schedules of the building.
\item Supply air static pressure -- Unless the precise location of the sensor and the overall duct layout is known, it is of little use in calculating the fan power. Does provide an indication of whether the fan is on or off which is useful for determining AHU schedules.
\item Various Setpoints -- In a well-controlled building, ideally, the value of the controlled sensor will be equivalent to the setpoint. However, the setpoint trends may not represent reality, especially under conditions of control overrides. 
\item Return air relative humidity -- May be an indicator of the level of latent load in the building. If the return air absolute humidity levels are relatively dry, the latent load may be zero or negligible. 
\item Outdoor air temperatures -- In some circumstances, the temperature
    of the fresh air may differ from the bulk air temperature in the
    local community, in which the local measurement will be a better
    indicator of the temperature of the air that is entering the AHU
    from the outdoors. However, the local weather station measurements
    are typically much more reliable and trustworthy than the sensors
    maintained at the site, and this needs to be taken into
    consideration.
\end{itemize}

Sensors that are commonly trended are not useful in estimating the energy use breakdown of a building:
\begin{itemize}
\item Return air CO\(_2\) -- Not directly related to measuring the energy use.
\item Damper commands -- Since duct layout and fluid flow models are not feasible, the damper commands do not provide information towards the calculation of the energy use. 
\item Chilled water valve/Hot water valve -- It is the actuator for controlling the supply air temperature, but does not provide information related to energy use.
\end{itemize}





