\chapter{\uppercase{Conclusions}} 

The following conclusions can be derived from this work.

\begin{enumerate}
    \item At the current time, sensor uncertainty and missing sensors
        reduces the potential effectiveness of an approach described.
    \item The mixed air temperature can be predicted to within X
    \item With no plenum temperature sensor, for terminal units that
        bring in plenum air, it was not possible to predict the plenum
        temperature in a useful manner. Assuming a constant value or the
        temperature of the corresponding space were not acceptable for
        this work. 
    \item The estimated zone load based on terminal unit flow and
        discharge temperature will depend on the sensitivity of the
        controls at the terminal unit. Hunting is present, which makes
        it difficult to gauge what the true zone load in the space is. 
    \item Setup in the prototype approach is fast and extensible to
        future buildings and system types. 
    \item Certain sensors should be prioritized over others. Suggestions
        are given in Section \ref{sec:Prioritization}. 
\end{enumerate}
