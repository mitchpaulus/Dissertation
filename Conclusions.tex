\chapter{\texorpdfstring{\MakeUppercase{Conclusions and Future Work}}{Conclusions and Future Work}}

The following conclusions can be derived from this work.

\begin{enumerate}
    \item Setup in the prototype approach is fast and extensible to
        future buildings and system types and has been implemented into
        the software tool \textit{Implementer}.
    \item Savings are highly dependent on the current operation. For two
        of the second floor air handlers in the NCTM building, no
        savings were to be found. However, in AHU-2-3, over 20\% energy
        savings were possible.
    \item The nearest neighbor method for predicting zone loads and the
        mixed air temperature was stable with regards to the chosen
        thresholds. The models tended to have better results with looser
        thresholds, meaning more recent data was a better predictor than
        data under closer conditions.
    \item Certain sensors should be prioritized for trending over
        others. Suggestions for commissioning agents are given in
        Section \ref{sec:Prioritization}.
    \item The mixed air temperature was able to be predicted within
        \SI{2}{\degF} for 5 of the 6 AHUs at NCTM (See \tableref{}
        \ref{tab:MATPredictionResultsForRun6}). This provides evidence
        that the nearest neighbor approach is viable for predicting the
        mixed air temperature.
    \item With no plenum temperature sensor, for terminal units that
        bring in plenum air, it was difficult to predict the plenum
        temperature in a useful manner. Assuming a constant value for the
        temperature of the corresponding space was not acceptable for
        this work with regards to estimating the reheat at the terminal units.
    \item The estimated zone load based on terminal unit flow and
        discharge temperature will depend on the sensitivity of the
        controls at the terminal unit. Hunting is present, which makes
        it difficult to gauge what the true zone load in the space is.
    \item The difference between the maximum measured flow and the design flow
        from the terminal units was under \SI{200}{\CFM} for all but one
        terminal unit at NCTM. This provided confidence in the
        possibility of using manufacturers specification in setting up
        the system.
    \item At the current time, missing sensors and sensor uncertainty
        reduces the potential effectiveness of an approach described.
        Additional sensors, especially those related to duct air flow,
        would improve the capabilities of the methodology.
\end{enumerate}

There are also areas for future work on this topic.

\begin{enumerate}
    \item With additional sensors related to the fans and air flow, in
        particular, fan power and pressure rise across the fan,
        modification of the supply air static pressure setpoint can be
        included in the optimization.
    \item Development of a prototype script to implement the logic on an
        existing BAS.
    \item Application to more potential case study buildings. There are
        numerous different styles of terminal units (single duct with
        and without reheat, parallel fan powered, dual-duct, bypass,
        induction, and underfloor) which all need
        further investigation. 
    \item Development of the Application Programming Interface (API)
        that would be used to interact with the different BAS systems, allowing
        these methods to scale effectively to thousands of buildings. 
\end{enumerate}

