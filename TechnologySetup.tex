\chapter{\texorpdfstring{\MakeUppercase{Technology Setup}}{Technology Setup}}

A major driving principle of the presented work is that the methodology
needs to be scalable. One reason for difficulties in implementing fault
detection and diagnostic algorithms and other optimization methods is
that there is a significant amount of initial investment necessary, in
time, risk, and money. 

In more complex schemes, it may be necessary to install additional
sensors that are typically not available in commercial HVAC systems.
Along with the sensor cost, there is the cost to configure the sensors
into the BAS. The updated logic will likely be coded by a controls
contractor, adding more cost. On a large campus, this reprogramming can
also take a significant amount of time to complete. 

Another issue is that of risk. There is the risk that if the controls do
not function properly and are too complex for the current building
operator, the controls cannot be easily removed and set back to the 
previous state. 


\section{Proposed Setup}

In the proposed setup, the change to the existing BAS is
minimal. The control logic remains the same. A small executable
script can be installed on the BAS computer that can host code that will
send an HTTP request to a remote server holding the historical data and
the optimization methods. 


\figref{} \ref{fig:NetworkFlow} shows potential information flow. Over
any period, historical trend data coming from the building or buildings
are stored in a server.  This data sync may only happen once, weekly,
daily or even sub-daily.  Once enough data has been stored on the
server, a client BAS computer (potentially one of many) can make an HTTP
request that will essentially carry information related to the current
time and what equipment the request corresponds to. Each AHU will have a
unique identifier that the main web server understands. 

The web server will implement the methods described in this document and
will send a response back to the client with the \textit{setpoint
values} for the system that will be near-optimal in a steady-state
sense, given the expected conditions at the building and individual
zones. 

\subsection{Standards for Transmission}

Standards for the format of information transfer are critical for rapid
adoption. An Application  Program Interface (API) is the public
interface of methods and routines that allow third-party programmers to
develop programs from the base building blocks. 

Another important point of consideration is the format of the data being
transferred. In the software industry, there are several types of data
exchange formats, two popular ones being JavaScript Object Notation
(JSON) and Extensible Markup Language (XML). The advantage of JSON over
XML is that JSON requires fewer characters to describe simple objects,
though, it does not support explicit schema definition. 

\begin{figure}
\begin{tikzpicture}

\node [anchor=center,draw] (BASComputer) at (0,0)
    {\includegraphics[width=3cm, height=3cm]{Images/DesktopComputerMPEdits.png}};

\node [anchor=center,draw] (Server) at (7cm, 0) {\includegraphics[width=3cm, height=3cm]{Images/serverImage.jpg}};

\draw [->, out=25, in=155, thick,anchor=center ] (BASComputer) to node
    [yshift=1cm, align=center] {HTTP request \\ \(\approx\)5-15 min} (Server);

\node [above=of Server] {Web Server};

\node [right=of Server, align=left, xshift=-5pt] {Historical data at \\ any syncing interval};

\draw [->, out=225, in=315, thick, ] (Server) to node [yshift=1cm, align=center] {Optimal Setpoints} (BASComputer);



\end{tikzpicture}    
\caption{Diagram of networking flow.}
\label{fig:NetworkFlow}
\end{figure}


\subsection{Advantages to Proposed System}

There are several key advantages to a system similar to the proposed:

\begin{enumerate}
    \item No real-time data transmission. Managing the networking of
        large BAS systems is difficult enough as is. It is not feasible
        for a single server to handle live streams of BAS point
        information from thousands of buildings. 

    \item The system can be added or removed quickly and without
        side-effects. Because the logic lives at an abstraction level
        above the BAS code, no changes to the system need to be made
        locally. 

    \item Routines only have to be added once for different equipment
        and setup types, allowing the same methods to expand to many
        different buildings. 
\end{enumerate}


\section{Prototype Information}

The proposed system was implemented in the software tool
\textit{Implementer}, a web application developed at the Energy Systems
Laboratory at Texas A\&M University. Implementer can analyze trend data
from any system that has an ability to store timestamp-value pairs and
allow access to them.  Various terms specifically related to Implementer are described in
\tableref{} \ref{tab:ImplementerTerms}.  

The current systems supported by Implementer include:

\begin{itemize}
    \item Alerton
    \item Andover Continuum
    \item Automated Logic
    \item Delta Controls Historian
    \item Emerson Ovation
    \item Honeywell Hoboware
    \item Johnson Controls Metasys
    \item Reliable Controls
    \item Siemens APOGEE
    \item TAC I/NET Seven
    \item TAC I/A Series
    \item Trane Tracer
    \item Tridium Niagara 
\end{itemize}

Before this process can begin, the sensors need to be mapped in a
way that the server can understand. 

\begin{table}
\centering
\caption{Terms specific to Implementer.}
\label{tab:ImplementerTerms}

\begin{tabular}{lp{4.9in}}
\toprule
    Term    & Description                                                                                      \\  \midrule
    Project & Information typically related to a collection of buildings                                       \\
    Label   & A known sensor type that can be associated with a trend, also called a tag by other software     \\
 Container  & An item that has a collection of trends associated,
 usually corresponding to a piece of equipment. Containers are set up into a tree
 structure with parent-child relationships, for example, a container
 representing an AHU would be a \textit{parent} container to a terminal
 unit. \\ \bottomrule
\end{tabular}

\end{table}

In Implementer, trends are given meaning through the use of
\textit{Labels}. Labels are similar to the concept of tagging that is used
in other systems, such as a system that implements Project Haystack. 

Implementer stores data in \textit{projects}. A project typically is a
collection of buildings from a campus, but could be a single building. 

Trends can receive labels in several ways. For trends in air handling
units, the label can automatically be derived from the type of sensor
and location on a schematic. A screenshot of a sample schematic from the
NCTM project is shown in \figref{} \ref{fig:AHUSchematic}. The different
types of equipment such as fans, coils, and temperature sensors are
placed on the schematic and are associated with an actual sensor. For
example, a temperature sensor that is at the outlet of the air handling
unit, unaffected by any other temperature affecting device, is defined
to have the \textit{supply air temperature} label. 

Labels can also be created manually. \figref{} \ref{fig:CustomLabels}
shows a screenshot of how all the real \textit{AUX TEMP} trends from the
different fan powered VAV (FPVAV) units at the case study NCTM building
are labeled as \textit{DischargeTemps}.

An entire site or building is organized into a hierarchy of components.
Each component is related to a \textit{container}, which trends can be
associated with.  A building container is a parent to air handling unit
containers, and air handling unit containers are parents to terminal
unit containers.  \figref{} \ref{fig:ContainerHierarchy} shows a
screenshot of a portion of the container hierarchy for the NCTM project.
Each line in \figref{} \ref{fig:ContainerHierarchy} is a different
container. The indentation indicates the relationships. By organizing
trends into containers which have parent-child relationships, along with
labeling, automated algorithms and analysis can easily be set up.

\begin{figure}
\centering
\includegraphics[scale=0.5]{Images/SampleAHUSchematic.PNG}
\caption{Screenshot of a sample AHU schematic for AHU-2-2 in the NCTM building in Implementer.}
\label{fig:AHUSchematic}
\end{figure}

\begin{figure}
\centering
\includegraphics[scale=0.75]{Images/CustomLabels.PNG}
\caption{Screenshot of the custom labels setup in Implementer, showing how all \textit{AUX TEMP} trends are associated to the label \textit{DischargeTemps}. }
\label{fig:CustomLabels}
\end{figure}

\begin{figure}
\centering
\includegraphics{Images/ContainerHierarchy.PNG}
\caption{Screenshot of the container hierarchy for NCTM in Implementer.}
\label{fig:ContainerHierarchy}
\end{figure}

To execute the optimization, the inputs and functions for the
various trends need to be set up. It is proposed that the use of a
container hierarchy, labels or tags, and custom equations be used to
develop the system.

Implementer uses something called container properties, and one of the
properties can be set to a JSON object that contains all the necessary
pointers to values or custom equations.

The necessary inputs for the AHU container would be 
\begin{enumerate}
    \item Prediction function for \(\mat{}\) 
    \item Design AHU flow 
    \item Exponent for the fan curve, \(n\)
    \item Fan curve constant, \(A\)
    \item Design fan power \(\dot{W}_{des}\)
    \item Maximum supply humidity ratio
    \item Minimum outdoor air flow percent
\end{enumerate}
For the terminal units, if a series fan powered
terminal unit is assumed, the necessary inputs will be
\begin{enumerate}
    \item Design flow
    \item Minimum flow percent
    \item Prediction function for zone temperature setpoint.
    \item Prediction function for zone load.
    \item Prediction function for plenum temperature.
\end{enumerate}

Parameters needed for the optimization are input for each container in
the container details portion of Implementer. The settings are formatted
in JSON, as shown in \figref{} \ref{fig:JSONOptions}. The properties
include options such as pointers to how the mixed air temperature
should be calculated, the maximum supply humidity ratio, and the
assumed part load ratio fan exponent. 

These pointers can be a static value, another property, an actual
trend from the building automation system, or a custom equation built up
using the powerful functions available in Implementer.

\begin{figure}
\centering
\begin{lstlisting}[language=json]
OptSATGenPoint = 
{"MATPrediction":"MAT Prediction",
"VDesign":"VDesign",
"FanExponent":2,
"WDesign":"WDesign",
"MaxHumidityRatio":0.009,
"Name":"Optimal SAT-Ex-2",
"BoundTermUnitProp":"SeriesBoxOptions",
"SATEnergySelector":"SAT"};
\end{lstlisting}
\caption{JSON options for the second floor air handling units. }
\label{fig:JSONOptions}
\end{figure}

Any calculation from the optimization analysis can be output as a trend
in Implementer and can be plotted and visualized using any of the tools
available. 

The equipment can be separated by project, which typically a building or
collection of buildings, and a call such as
\url{https://domain/GetEquipmentForProject?ProjectId={Project-Id}}
could return all the possible set up equipment.

The interface for receiving the optimal setpoints could have a URL
similar to 
\url{https://domain/setpoints?id={Equipment-Id}\&{time=Datetime}}




