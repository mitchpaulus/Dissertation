%%%%%%%%%%%%%%%%%%%%%%%%%%%%%%%%%%%%%%%%%%%%%%%%%%%
%
%  New template code for TAMU Theses and Dissertations starting Fall 2012.  
%  For more info about this template or the 
%  TAMU LaTeX User's Group, see http://www.howdy.me/.
%
%  Author: Wendy Lynn Turner 
%	 Version 1.0 
%  Last updated 8/5/2012
%
%%%%%%%%%%%%%%%%%%%%%%%%%%%%%%%%%%%%%%%%%%%%%%%%%%%

%%%%%%%%%%%%%%%%%%%%%%%%%%%%%%%%%%%%%%%%%%%%%%%%%%%%%%%%%%%%%%%%%%%%%%
%%                           APPENDIX A 
%%%%%%%%%%%%%%%%%%%%%%%%%%%%%%%%%%%%%%%%%%%%%%%%%%%%%%%%%%%%%%%%%%%%%

\phantomsection

\chapter{\texorpdfstring{\MakeUppercase{Approximations to the Ramp Function}}{Approximations to the Ramp Function}}

\section{Approximation Using Fourier Series}

This section shows a derivation of the Fourier Series for the ramp
function, which is often used to model the output flow for a terminal
unit.

In many cases related to steady state algorithms for determining the
energy use of an air handling unit, ramp functions are found. These ramp
functions are usually difficult to deal with in optimization problems
because they are not continuously differentiable. In this sense, it is
desired to have a replacement or approximation to the function that is
smooth and can be differentiated. 

Fourier series are a useful tool for approximating arbitrary functions
and can be used in this task. We can begin by focusing on the basic ramp
function that is defined by:

\begin{equation}
f(x)=\begin{cases}0  & x < 0 \\
               x  & x \geq 0
\end{cases}
\end{equation}

If we define our Fourier series to be defined over the interval \(-L
\leq x \leq L\) and be equal to 

\begin{equation}
 f(x) = a_0 + \sum_{n=1}^\infty a_n \cos{\frac{n \pi x }{L}} + \sum_{n=1}^\infty b_n \sin{\frac{n \pi x }{L}}
\end{equation}

The coefficients are equal to

\begin{equation}
 a_0 = \frac{1}{2 L } \int_{-L}^{L} f(x) \; dx 
\end{equation}


\begin{equation}
 a_n = \frac{1}{L} \int_{-L}^{L} f(x) \cos{\frac{n \pi x}{L}} \; dx 
\end{equation}

\begin{equation}
 b_n = \frac{1}{L} \int_{-L}^{L} f(x) \sin{\frac{n \pi x}{L}} \; dx 
\end{equation}

\(f(x)=0\) when \(x \leq 0\), so that portion of the integral is equal to 0. When \(x \geq 0\), \(f(x)=x\) and the coefficients can be evaluated over the range \(0 \leq x \leq L\).

\begin{equation}
 a_0 = \frac{1}{2 L } \int_{0}^{L} x \; dx 
\end{equation}


\begin{equation}
 a_n = \frac{1}{L} \int_{0}^{L} x \cos{\frac{n \pi x}{L}} \; dx 
\end{equation}

\begin{equation}
 b_n = \frac{1}{L} \int_{0}^{L} x \sin{\frac{n \pi x}{L}} \; dx 
\end{equation}


Solving for \(a_0\), 
%
\begin{equation}
 a_0 = \frac{1}{2 L } \int_{0}^{L} x \; dx = \frac{1}{2 L} \left[ \frac{x^2}{2} \right]^{L}_{0} = \frac{1}{2 L} \left(\frac{L^2}{2} \right) = \frac{L}{4}
\end{equation}
%
The coefficients for the cosine terms are 
%
\begin{equation}
 \begin{split} 
     a_n &= \frac{1}{L} \int_{0}^{L} x \cos{\frac{n \pi x}{L}} \; dx \\ 
         &=  \frac{1}{L} \left(\cancelto{0}{x \frac{L}{n \pi}   \sin{\frac{n \pi x}{L}}  \bigg|^{L}_0}  - \int_0^L \frac{L}{n \pi} \sin{\frac{n \pi}{L} x} \; dx  \right) \\
         &= \frac{1}{L} \left( \left[ \frac{L^2}{n^2 \pi^2} \cos{\frac{n \pi x}{L}}  \right]^L_0 \right) \\
         &= \frac{L}{n^2 \pi^2}\left( (-1)^n - 1 \right)
 \end{split}
\end{equation}
%
%
and the coefficients for the sine terms are 
%
%
\begin{equation}
 \begin{split} 
     b_n &= \frac{1}{L} \int_{0}^{L} x \sin{\frac{n \pi x}{L}} \; dx \\ 
         &=  \frac{1}{L} \left(-x \frac{L}{n \pi}   \cos{\frac{n \pi x}{L}}  \bigg|^{L}_0  - \int_0^L -\frac{L}{n \pi}\cos{\frac{n \pi}{L} x} \; dx  \right) \\
         &= \frac{1}{L} \left( \left[ \frac{-L^2}{n \pi} (-1)^n  +\frac{L^2}{n^2 \pi^2}   \left[ \sin{\frac{n \pi x}{L}} \right]_0^L  \right] \right) \\
         &= \frac{1}{L} \left( \left[ \frac{-L^2}{n \pi} (-1)^n  + 0 \right] \right) \\
         &= \frac{L}{n \pi}\left( (-1)^{n+1} \right)
 \end{split}
\end{equation}
%\begin{figure}[H]
%\centering
%\includegraphics[scale=.50]{figures/Penguins.jpg}
%\caption{TAMU figure}
%\label{fig:tamu-fig5}
%\end{figure}

\section{Approximation Using the Logistic Function}

The threshold function is the derivative of the ramp function. The
threshold function can be approximated arbitrarily well using the
logistic function. The logistic function has the form
\begin{equation}\label{eq:ThresholdFunction}
    y=\frac{1}{1+e^{-k\left(x-x_{0}\right)}}  
\end{equation}
As \(k\to\infty\), the function equals the threshold function. The
integral of the threshold function is the ramp function. Integrating 
Eq. \ref{eq:ThresholdFunction} results in the family of solutions
\begin{equation}
    y=x+\frac{\ln \left(1+e^{-k\left(x-x_{0}\right)}\right) }{k} + C
\end{equation}
When \(x > x_{0}\), the \(\ln \left(1+e^{-k\left(x-x_{0}\right)}\right)
\) term goes to 0, and the function is equal to \(x+C\), which has a
derivative of 1 with respect to \(x\).

When \(x < x_{0}\), the \(\frac{\ln \left(1+e^{-k\left(x-x_{0}\right)}\right) }{k}\) term goes to \(x_{0} - x \), and the function is equal to 
\begin{equation}
    y = x + \left(x_{0} - x \right) + C = x_{0} + C
\end{equation}

This function approximates the ramp function arbitrarily well as
\(k\to\infty\). 

The function is usually applied to flow for terminal units with a
minimum flow setting. In this case, the value for the constant \(C\)
equals 0 so that \(x_{0}\) and \(x_{0}  +C \) are equal, and the function becomes

\begin{equation}
    \flow{} = \flow{req} + \frac{\ln{} \left(1+e^{-k\left( \flow{req} - \flow{min}
    \right)}\right)}{k} 
\end{equation}


