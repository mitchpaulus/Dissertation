%%%%%%%%%%%%%%%%%%%%%%%%%%%%%%%%%%%%%%%%%%%%%%%%%%%
%
%  New template code for TAMU Theses and Dissertations starting Fall 2012.  
%  For more info about this template or the 
%  TAMU LaTeX User's Group, see http://www.howdy.me/.
%
%  Author: Wendy Lynn Turner 
%	 Version 1.0 
%  Last updated 8/5/2012
%
%%%%%%%%%%%%%%%%%%%%%%%%%%%%%%%%%%%%%%%%%%%%%%%%%%%

%%%%%%%%%%%%%%%%%%%%%%%%%%%%%%%%%%%%%%%%%%%%%%%%%%%%%%%%%%%%%%%%%%%%%%
%%                           SECTION I
%%%%%%%%%%%%%%%%%%%%%%%%%%%%%%%%%%%%%%%%%%%%%%%%%%%%%%%%%%%%%%%%%%%%%


\pagestyle{plain} % No headers, just page numbers
\pagenumbering{arabic} % Arabic numerals
\setcounter{page}{1}


\chapter{\texorpdfstring{\MakeUppercase{Introduction}}{Introduction}} 
Adjusting the HVAC control sequences in existing building commissioning
is a standard method to reduce energy consumption. Many of the original
control sequences for building equipment are never optimized or adjusted
using detailed engineering, due not only to the amount of time and
effort that such an analysis may take but also due to the lack of skill
the on-site maintenance staff may have. 

Sensor data from building automation systems are becoming more abundant
as computing resources decrease in price and software improves in
quality.  Software applications can use this wealth of information in an
automated process to actively optimize the air conditioning system,
without detailed input from an engineer. 

This work attempts to leverage commonly available trend data to optimize the
setpoint values that air handling units of single duct variable air volume
systems typically use. Ideally, to be considered optimal, the entire airside
system needs to be considered as a whole, including fan energy, cooling energy,
and reheat energy. While demand-based controls can often significantly reduce
energy use for one component of the system, it does not necessarily optimize
the whole. 

It is desired to refrain from adjusting the existing control logic or low-level
electronics to accomplish this outcome. This work attempts to acquire trend
data, run the necessary methods to determine the optimal setpoints from a
separate dedicated system, and then send the information back and
\textit{actively} change the air handling unit setpoints in the BAS. In this
way, the methodology can scale quickly to many different air handlers and
buildings, being indifferent to the vendor of the BAS.
