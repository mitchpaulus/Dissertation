%%%%%%%%%%%%%%%%%%%%%%%%%%%%%%%%%%%%%%%%%%%%%%%%%%%
%
%  New template code for TAMU Theses and Dissertations starting Fall 2012.  
%  For more info about this template or the 
%  TAMU LaTeX User's Group, see http://www.howdy.me/.
%
%  Author: Wendy Lynn Turner 
%	 Version 1.0 
%  Last updated 8/5/2012
%
%%%%%%%%%%%%%%%%%%%%%%%%%%%%%%%%%%%%%%%%%%%%%%%%%%%
%%%%%%%%%%%%%%%%%%%%%%%%%%%%%%%%%%%%%%%%%%%%%%%%%%%%%%%%%%%%%%%%%%%%%
%%                           ABSTRACT 
%%%%%%%%%%%%%%%%%%%%%%%%%%%%%%%%%%%%%%%%%%%%%%%%%%%%%%%%%%%%%%%%%%%%%

\chapter*{\texorpdfstring{\MakeUppercase{ABSTRACT}}{ABSTRACT}}
\addcontentsline{toc}{chapter}{ABSTRACT} % Needs to be set to part, so the TOC doesnt add 'CHAPTER ' prefix in the TOC.

\pagestyle{plain} % No headers, just page numbers
\pagenumbering{roman} % Roman numerals
\setcounter{page}{2}

\indent  In this work, a new paradigm for the optimization of air
handling unit systems in buildings was explored. The methods assume
that only commonly trended sensor data would be available, and that no
live connection to trend values existed. An actual implementation would
only require a small script to be written at the target building to
request information from a centralized server. 

A prioritization of sensors to trend at buildings is presented.
Investigations in the methodology possibilities were completed on a case study
building on the Texas A\&M campus, the National Center for Therapeutic
Medicine (NCTM). The algorithms and models for the optimization are
presented, along with uncertainty analysis into several key model
parameters. 

Approximately 23-29\% energy savings were found for AHU-2-3 at the NCTM
building from June 1\textsuperscript{st} to, 2016 to January
1\textsuperscript{st}, 2017.  Missing fan power and air flow sensors were a significant
determent to the method, along with uncertainty in the plenum
temperature for the series fan powered terminal units. Lack of easily
accessible, accurate, manufacturers specifications were also
limitations.  

A prototype of the backend system was developed on the web application
\textit{CC-Compass}, available at Texas A\&M. The system is set up with
a parent-child hierarchy for equipment and equipment properties are
specified in a JSON format. 
 

\pagebreak{}
