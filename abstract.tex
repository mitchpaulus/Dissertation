%%%%%%%%%%%%%%%%%%%%%%%%%%%%%%%%%%%%%%%%%%%%%%%%%%%
%
%  New template code for TAMU Theses and Dissertations starting Fall 2012.  
%  For more info about this template or the 
%  TAMU LaTeX User's Group, see http://www.howdy.me/.
%
%  Author: Wendy Lynn Turner 
%	 Version 1.0 
%  Last updated 8/5/2012
%
%%%%%%%%%%%%%%%%%%%%%%%%%%%%%%%%%%%%%%%%%%%%%%%%%%%
%%%%%%%%%%%%%%%%%%%%%%%%%%%%%%%%%%%%%%%%%%%%%%%%%%%%%%%%%%%%%%%%%%%%%
%%                           ABSTRACT 
%%%%%%%%%%%%%%%%%%%%%%%%%%%%%%%%%%%%%%%%%%%%%%%%%%%%%%%%%%%%%%%%%%%%%

\chapter*{\texorpdfstring{\MakeUppercase{ABSTRACT}}{ABSTRACT}}
\addcontentsline{toc}{chapter}{ABSTRACT} % Needs to be set to part, so the TOC doesnt add 'CHAPTER ' prefix in the TOC.

\pagestyle{plain} % No headers, just page numbers
\pagenumbering{roman} % Roman numerals
\setcounter{page}{2}

\indent  In this work, a new concept was explored for the optimization
of heating, ventilating, and air-conditioning (HVAC) systems in
buildings. The methods assume that only commonly trended sensor data
would be available and that no live connection to sensor values would
exist.  An actual implementation would only require a small script to be
written at the target building to request information from a centralized
server and update setpoint values. 

A prioritization of sensors to trend at buildings is presented.
Investigations into the feasibility were completed on a case study
building on the Texas A\&M Campus, the National Center for Therapeutic
Medicine (NCTM) and the Preston Royal Library. The algorithms and models
for the optimization are presented, along with uncertainty analysis into
several key model parameters. 

23-29\% energy savings were found for AHU-2-3 at the NCTM
building from June 1\textsuperscript{st}, 2016 to January
1\textsuperscript{st}, 2017.  Missing fan power and air flow sensors
reduced effectiveness, along with uncertainty in the plenum temperature
for the series fan powered terminal units. Lack of readily available,
accurate, manufacturers specifications were also limitations.  

A prototype of the system was developed on the web application
\textit{CC-Compass}, available at Texas A\&M. 

\pagebreak{}
