\section{Advanced Computation Techniques and Controls}

The HVAC industry has slowly begun to employ the advances in computer
science. Researchers from the University of Iowa (Kusiak, Li, Xu, Tang,
Wei) have published a great deal of work on the optimization of all
types of HVAC systems. They have applied data mining algorithms and
computational intelligence algorithms to data-driven optimization for
the cooling output of air handling units and plants
\cite{Kusiak2014MinimizationOfEnergyConsumptionInHVAC, HeXiaofei2014,
Kusiak2013MinimizingEnergyConsumption,
Kusiak2012ModelingAndOptimizationOfHVAC, Kusiak2011MultiObjective,
Kusiak2010ReheatBox, WeiXiupeng2015,
WeiXiupeng2014ModelingAndOptimizationOfAChillerPlant, Kusiak2010,
Kusiak2010ModelingAndOptimization,
Kusiak2011OptimizationOfAnHVACSystemWithAStrength}. They have used
techniques like neural networks, evolutionary type programming, and
multi-perceptron ensembles. They have also applied
\textit{particle-swarm optimization} in several of their publications. 

The use of genetic algorithms or general evolutionary programming
techniques has been applied in several pieces of research. Genetic
algorithms have been used to optimize chilled water supply temperature,
supply air temperature, fan control, and outdoor air control, related to
many different kinds of systems, including variable air volume and
variable refrigerant volume
\cite{Fong2006HVACProgramming,Jin2005Prediction-basedSystems,Parameshwaran2010EnergyAlgorithm,Congradac2009HVACAlgorithms}

\subsection{Model Predictive Control}

Model predictive (or receding-horizon) control (MPC) has been a popular
research field in control theory and has been successfully implemented
in practice. Li et. al \cite{Li2015} recently showed the
benefits of MPC in both simulation and in experimental work. They
estimated electrical consumption savings to be 18\% for a 75,150
ft\(^2\) building in Philadelphia during a week in August, and found
that in 75\% of their 20 test days they had energy consumption savings
of over 20\%. They also used a centralized architecture where BAS data
were passed through a middleware with a historical database to Matlab and
an AMPL optimization system, with results of that system dynamically changing
the building HVAC system.

% The citations here are coming directly from Afram 2017.
Afram et al. also studied the combination of ANNs and MPC
\cite{Afram2017}. As described in \cite{Afram2017}, the combination of
these two techniques has been used for the following different control
objectives:

\begin{enumerate}
    \item Minimize energy consumption \cite{Ferreira2012, Huang2015a, Kusiak2011OptimizationOfAnHVACSystemWithAStrength, Kusiak2014MinimizationOfEnergyConsumptionInHVAC,WeiXiupeng2015, Garnier2015, Kim2016, LuLu2005HVACSystemOptimization, Ning2010Neuro-optimalSystem}
    \item Maintain thermal comfort \cite{Ferreira2012, Kusiak2011OptimizationOfAnHVACSystemWithAStrength, Kusiak2014MinimizationOfEnergyConsumptionInHVAC, WeiXiupeng2015, Garnier2015, Kim2016}
    \item Maintain indoor air quality (IAQ) at an acceptable level \cite{Kusiak2011MultiObjective}
    \item Minimize operating cost \cite{Garnier2015, Lee2015, Huang2015a, Ruano2015, Ruano2016}
    \item Maintain visual comfort at an acceptable level \cite{Kim2016}
    \item Minimize retrofit cost \cite{Asadi2014a}
    \item Minimize thermal discomfort hours \cite{Asadi2014a}
\end{enumerate}

This dissertation has a focus on steady state behavior, but the dynamic
behavior of buildings and its controls are also important. Seem has
published research comparing a finite state machine (FSM) sequencing to
the more common split-range sequencing control logic \cite{Seem1999}.
Xu, Li, and Cai proposed a receding-horizon optimization control that
uses a typical PID type controller \cite{XuMin2005}. This work had a
focus on practicability in that it required no changes in the hardware
or  the definitions of the common control parameters related to a PID
controller. Yuan and Perez used a model-predictive controller to control
temperature and ventilation for multiple zones
\cite{Yuan2006Multiple-zoneStrategy}. Freire, Oliveira, and Mendes also
used predictive controllers for thermal comfort optimization
\cite{Freire2008PredictiveSavings}.  Guo, Song, and Cai investigated
neural networks in HVAC control \cite{Guo2007}.
